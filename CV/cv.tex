% LaTeX Curriculum Vitae Template
%
% Copyright (C) 2004-2009 Jason Blevins <jrblevin@sdf.lonestar.org>
% http://jblevins.org/projects/cv-template/
%
% You may use use this document as a template to create your own CV
% and you may redistribute the source code freely. No attribution is
% required in any resulting documents. I do ask that you please leave
% this notice and the above URL in the source code if you choose to
% redistribute this file.

\documentclass[letterpaper]{article}

\usepackage{hyperref}
\usepackage{geometry}
\usepackage{tabularx}
\usepackage{longtable}
\usepackage{hyperref}
\usepackage{pbox}

\newenvironment{blockquote}{%
	\par%
	\medskip
	\leftskip=4em\rightskip=2em%
	\noindent\ignorespaces}{%
	\par\medskip}

% Comment the following lines to use the default Computer Modern font
% instead of the Palatino font provided by the mathpazo package.
% Remove the 'osf' bit if you don't like the old style figures.
\usepackage[T1]{fontenc}
\usepackage[sc,osf]{mathpazo}

% Set your name here
\def\name{Vencislav Popov}

% Replace this with a link to your CV if you like, or set it empty
% (as in \def\footerlink{}) to remove the link in the footer:
%\def\footerlink{http://jblevins.org/projects/cv-template/}

% The following metadata will show up in the PDF properties
\hypersetup{
  colorlinks = true,
  urlcolor = black,
  pdfauthor = {\name},
  pdftitle = {\name: Curriculum Vitae},
  pdfsubject = {Curriculum Vitae},
  pdfpagemode = UseNone
}

\geometry{
  body={7.25in, 9.5in},
  left=0.75in,
  top=0.75in
}

% Customize page headers
\pagestyle{myheadings}
\markright{\name}
\thispagestyle{empty}

% Custom section fonts
\usepackage{sectsty}
\sectionfont{\rmfamily\mdseries\Large}
\subsectionfont{\rmfamily\mdseries\itshape\large}

% Other possible font commands include:
% \ttfamily for teletype,
% \sffamily for sans serif,
% \bfseries for bold,
% \scshape for small caps,
% \normalsize, \large, \Large, \LARGE sizes.

% Don't indent paragraphs.
\setlength\parindent{0em}

% Make lists without bullets
\renewenvironment{itemize}{
  \begin{list}{}{
    \setlength{\leftmargin}{1.5em}
  }
}{
  \end{list}
}

\begin{document}

% Place name at left
{\huge \name}

% Alternatively, print name centered and bold:
%\centerline{\huge \bf \name}

%\vspace{0.25in}

%\begin{minipage}{0.45\linewidth}
%  \href{http://www.nbu.bg/}{New Bulgarian University} \\
%  Department of Statistics \\
%  Smith Building \\
%  Chapel Hill, NC 27599
%\end{minipage}
\subsubsection*{March, 2023}
\begin{minipage}{0.80\linewidth}
	Ven Popov  | vencislav.popov@gmail.com | http://venpopov.com\\
\end{minipage}


\section*{Education}


\begin{tabular}{ll}
	2020 & Ph.D., Psychology, Carnegie Mellon University\\
	& {\it Dissertation Title:}  Resource Depletion and Recovery in Human Memory\\
	& {\it Adviser:} Lynne M. Reder | Committee: John Anderson, Marlene Behrmann, Marc Coutanche\\
	& {\it Glushko Dissertation Prize (2021)}\\[6pt]
	2015 & B.A., Psychology, New Bulgarian University\\
\end{tabular}

\section*{Research Experience}


\begin{tabular}{ll}
      02.2023 - current & Senior Scientist in Computational Modeling (tenured)\\
      & Department of Psychology, University of Zurich \\[6pt]
	08.2020 - 01.2023 & Postdoctoral Researcher, Cognition Lab, University of Zurich \\[6pt]
	07.2015 - 06.2020 & Doctoral Research, Memory Lab, Carnegie Mellon University\\[6pt]
	01.2015 - 08.2015 & Data Analyst, Bulgarian Addictions Institute \\
\end{tabular}





\section*{Scholarships and Awards}
\begin{longtable}{rl}
	2023 & \textbf{Bertelson Early Career Award}\\
    & {\it "To an outstanding young scientist for making a significant contribution to European Cognitive Psychology"} \\
	& European Society for Cognitive Psychology\\[6pt]
    
 & \textbf{ESCOP Early Career Publication Award}\\
	& {\it "For outstanding work that is truly transformational for the field of human learning and memory"}\\
	& European Society for Cognitive Psychology\\[6pt]
	2021 & \textbf{Robert J. Glushko Dissertation Prize (\$10,000)}\\
	& {\it For an outstanding dissertation in cognitive science}\\
	& Cognitive Science Society \& Robert J. Glushko and Pamela Samuelson Foundation\\[6pt]
	2020 & \textbf{McClelland Outstanding Paper Award}\\
	& Center for the Neural Basis of Cognition, PA\\[6pt]
	& \textbf{Bobby Klatzky Graduate Student Publication Award}\\
	& {\it For outstanding performance in scholarly research and writing}\\
	& Department of Psychology, Carnegie Mellon University \\[6pt]
	2019 & \textbf{Bobby Klatzky Graduate Student Publication Award}\\
	& {\it For outstanding performance in scholarly research and writing}\\
	& Department of Psychology, Carnegie Mellon University \\[6pt]
	2017 & \textbf{Cognitive Science Society Travel Award} \\
	&   {\it For an oustanding student paper}\\
	& Robert J. Glushko and Pamela Samuelson Foundation\\[6pt]
	2015 - 2016 & \textbf{Presidential Fellowship in the Life Sciences}\\
	& Richard King Mellon Foundation\\[6pt]
	%2014 - 2015 & \textbf{Full academic scholarship}\\
%	& New Bulgarian University\\[6pt]
%	2014 & \textbf{Scholarship for attending ``European Campus of Excellence Summer School''}\\
%	& Stiftung Mercator\\[6pt]
%	2011 - 2013 & \textbf{Full academic scholarship}\\
%	& Sofia University 'St. Kliment Ohridsky'\\[6pt]
%	2013 - 2015 & \textbf{Multiple research excellency awards}\\
%	& European Social Fund, European Union \& The Ministry of Education and Science of Bulgaria\\[6pt]
%	2012 - 2015 & \textbf{Academic \& research excellence scholarships}\\
%	& European Social Fund, European Union \& The Ministry of Education and Science of Bulgaria\\[6pt]
\\
\end{longtable}

\section*{Publications} \rightline{\textbf{(* = co-first author; $\dagger$ = corresponding author; \underline{student co-author})}}

 \subsection*{Manuscripts in progress}

 \begin{longtable}{p{0.7cm}p{15cm}}
 & \textbf{Popov, V \& Oberauer, K.} (submitted).How to build an observatory for the mind\\[6pt]
&\\
& Trueblood, J. S., Allison, D., Field, S. M., Fishbach, A., Gaillard, S. D. M., Gigerenzer, G., … \textbf{Popov, V.},... Teodorescu, A. (submitted). The Misalignment of Incentives in Academic Publishing and Implications for Journal Reform. Preprint available at https://doi.org/10.31234/osf.io/jrczf\\[6pt]
&\\
& Binz, M., Alaniz, S., Roskies, A., Aczel, B., Bergstrom, C. T., Allen, C., … \textbf{Popov, V.} … Schulz, E. (submitted). How should the advent of large language models affect the practice of science?. Preprint available at https://doi.org/10.31219/osf.io/yr9xb\\[6pt]
&\\
 & \textbf{Popov, V.} (under review). Cognitive resources can be intentionally released when processed information becomes irrelevant: Insights from the primacy effect in working memory. Preprint available at https://psyarxiv.com/gct58\\[6pt]
&\\ 
& \textbf{Popov, V.} (under revision). If God Handed Us the Ground-Truth Theory of Memory, How Would We Recognize It? Preprint available at https://psyarxiv.com/ay5cm/\\[6pt]
&\\
& Ma, S., \textbf{Popov, V. \textsuperscript{\textdagger}}, \& Zhang, Q \textsuperscript{\textdagger}. (under revision).  A Neural Index Reflecting the Amount of Cognitive Resources Available during Memory Encoding: A Model-based Approach. Preprint available at https://www.biorxiv.org/content/10.1101/2022.08.16.504058v1\\[6pt]
&\\
& Frischkorn, G.* \& \textbf{Popov, V.* (co-first authors)} (under revision). A tutorial for estimating mixture models for visual working memory tasks in brms: Introducing the Bayesian Measurement Modeling (bmm) package for R. Preprint available at https://psyarxiv.com/umt57\\[6pt]
&\\
& Dames, H., Musfeld, P., \textbf{Popov, V.}, Oberauer, K., \& Frischkorn, G. (under review). Responsible Research Assessment Should Prioritize Theory Development and Testing Over Ticking Open Science Boxes. Preprint available at https://psyarxiv.com/ad74m/\\
 \end{longtable}

\subsection*{Journal publications}

\begin{longtable}{p{0.7cm}p{15cm}}
2023 & Dames, H. \& \textbf{Popov, V.} (in press). When does intent matter for memory? Bridging perspectives with Craik. {\it Journal of Experimental Psychology: General.}  Preprint available at https://psyarxiv.com/gwd4s/\\[6pt]
&\\
& \textbf{Popov, V.\textsuperscript{\textdagger}} \& \underline{Dames, H.} (2023). Intent Matters: Resolving the Intentional vs Incidental Learning Paradox in Episodic Long-term Memory. {\it Journal of Experimental Psychology: General, 152(1), 268–300}. Preprint available at https://psyarxiv.com/jf2en \vspace{4pt}\newline 
\-\hspace{0.8cm}- {\it 2023 ESCOP Early Career Pubclication Award} \\[6pt]
& \\
2022 & Norton, C. M., Ibinson, J. W., Pcola, S. J., \textbf{Popov, V.}, Tremel, J. J., Reder, L. M., ... \& Vogt, K. M. (2022). Neutral auditory words immediately followed by painful electric shock may show reduced next-day recollection. {\it Experimental Brain Research}, 1-13.\\[6pt]
& \\
2021 & \textbf{Popov, V.\textsuperscript{\textdagger}}, So, M., \& Reder, L. (2021). Memory resources recover gradually over time: The effects of word-frequency, presentation rate and list-composition on binding errors and mnemonic precision in source memory. {\it Journal of Experimental Psychology: Learning, Memory \& Cognition}\\[6pt]
&\\
& Vogt, K., Ibinson, J., Smith, C., Citro, A., Norton, C., Karim, H., \textbf{Popov, V.}, Mahajan, A., Aizenstein, H., Reder, L. \& Fiez, J. (in press). Midazolam and ketamine produce distinct neural changes in memory, pain, and fear networks during pain. {\it Anesthesiology.}\\[6pt]
& \\
	
2020 & \textbf{Popov, V.\textsuperscript{\textdagger}}, \& Reder, L. (2020). Greater discrimination difficulty during perceptual learning leads to stronger and more distinct representations. {\it Psychonomic Bulletin \& Review.} Advance Online Publication\\[6pt]
& \\ 

& Vassileva, J., Psedarska, E., Yankov, G., Bozgunov, K., \textbf{Popov, V.} \& Vasilev, G. (2020). Validation of the Levenson Self-Report Psychopathy Scale in Bulgarian Substance Dependent Individuals. {\it Frontiers in Psychology.} 11:1110 \\[6pt]
& \\ 

& \textbf{Popov, V.\textsuperscript{\textdagger}} \& Reder, L.\textsuperscript{\textdagger}(2020). Frequency effects on memory: A resource-limited theory. {\it Psychological Review. 127}(1), 1-46.\vspace{4pt}\newline 
\-\hspace{0.8cm}- {\it 2020 McClelland Outstanding Paper Award, CNBC}\newline
\-\hspace{0.8cm}- {\it 2020 Bobby Klatzky Graduate Student Publication Award, CMU} \\[6pt]
& \\ 

2019 & \textbf{Popov, V.\textsuperscript{\textdagger}}, Marevic, I.,  Rummel, J., \& Reder, L. (2019). Forgetting is a feature, not a bug: Intentionally forgetting some things helps us remember others by freeing up working memory resources. {\it Psychological Science. 30}(9), 1303-1317. \\[6pt]
& \\ 
% &{\it Pre-registered (https://osf.io/nz6cq). Open access materials, data, code
% (https://osf.io/5qd94/).}\\[8pt]
% &\\
& \textbf{Popov, V.\textsuperscript{\textdagger}}, Zhang, Q., Koch, G., Halloway, R. \& Coutanche, M. (2019). Semantic knowledge influences whether novel episodic associations are represented symmetrically or asymmetrically. {\it Memory \& Cognition. 47}(8), 1567-1581. \\[6pt]
& \\ 
% &{\it Pre-registered (https://osf.io/rdsw5). Open access materials, data, code
% (https://osf.io/72amw)}\\[8pt]
% &\\
2018 & \textbf{Popov, V.\textsuperscript{\textdagger}}, \underline{Ostarek, M.} \& \underline{Tenison, C.} (2018). Practices and pitfalls in inferring neural representations. {\it NeuroImage. 174}(1), 340-351\vspace{4pt}\newline 
\-\hspace{0.8cm}- {\it 2019 Bobby Klatzky Graduate Student Publication Award, CMU} \\[6pt]
 &\\
& Shen, Z.*, \textbf{Popov, V.* (co-first authors)}, Delahay, A., \& Reder, L. (2018). Item strength affects working memory capacity. {\it Memory \& Cognition. 46}(2), 204-215. \\[6pt]
 &\\
2017 & \textbf{Popov, V.\textsuperscript{\textdagger}}, Hristova, P., \& Anders, R. (2017). The Relational Luring Effect: Retrieval of relational information during associative recognition. {\it Journal of Experimental Psychology: General. 146}(5), 722-745\\[6pt]
 &\\
& Manelis, A.*\textsuperscript{\textdagger}, \textbf{Popov, V.*\textsuperscript{\textdagger} (co-first authors)}, Paynter, C., Walsh, M., Wheeler, M., Vogt, K., \& Reder, L.\textsuperscript{\textdagger} (2017). Cortical networks involved in memory for temporal order. {\it Journal of Cognitive Neuroscience. 29}(7), 1253-1266.\\[6pt]
&\\
2016 &  Reder, L. M., Liu, X. L., Keinath, A., \textbf{Popov, V.} (2016). Building knowledge requires bricks, not sand: The critical role of familiar 
constituents in learning. {\it Psychonomic Bulletin \& Review}. {\it 23}(1), 271-277.\\[6pt]
 &\\
2015 & \textbf{Popov, V.\textsuperscript{\textdagger}} \& Hristova, P. (2015). Unintentional and efficient relational priming. {\it Memory \& Cognition}, {\it 46}(6), 866-878.\\[6pt]
\end{longtable}


\subsection*{Book chapters}

\begin{longtable}{p{0.7cm}p{15cm}}
2019 & \textbf{Popov, V.} \& Reder, M. (In press). Frequency effects in recognition and recall. To appear In M. Kahana \& A. Wagner (Eds.), {\it Handbook on Human Memory.} Oxford University Press
\end{longtable}

\subsection*{Refereed full conference papers}

\begin{longtable}{p{0.7cm}p{15cm}}
2018  &  Zhang, Q.*, \textbf{Popov, V.* (co-first authors)}, Koch, G., Halloway, R. \& Coutanche, M. (2018). Fast Memory Integration Facillitated by Schema Consistency. In C. Kalish, M. Rau, J. Zhu, T. Rogers (Eds.), {\it Proceedings of the 40th Annual Conference of the Cognitive Science Society}. Austin, TX: Cognitive Science Society.\\[6pt]
&\\
2017 & \textbf{Popov V.}, Ostarek, M., \& Tenison, C. (2017). Inferential Pitfalls in Decoding Neural representations. In G. Gunzelmann, A. Howes, T. Tenbrink, \& E. Davelaar (Eds.), {\it Proceedings of the 39th Annual Conference of the Cognitive Science Society} (pp. 961-966). Austin, TX: CSS.\\[6pt]
 &\\
& \textbf{Popov V.}, \& Reder, L. (2017). Repetition improves memory by strengthening existing traces: Evidence from paired-associate learning under midazolam. In G. Gunzelmann, A. Howes, T. Tenbrink, \& E. Davelaar (Eds.), {\it Proceedings of the 39th Annual Conference of the Cognitive Science Society} (pp. 2913-2918). Austin, TX: Cognitive Science Society.\\[6pt]
 &\\
& \textbf{Popov V.}, \& Reder, L. (2017). Target-to-distractor similarity can help visual search performance. In G. Gunzelmann, A. Howes, T. Tenbrink, \& E. Davelaar (Eds.), {\it Proceedings of the 39th Annual Conference of the Cognitive Science Society} (pp. 968-973). Austin, TX: Cognitive Science Society.\\[6pt]
 &\\
2014 & \textbf{Popov, V.} \& Hristova, P. (2014). Automatic analogical reasoning underlies structural priming in comprehension of ambiguous sentences. In P. Bello, M. Guarini, M. McShane, \& B. Scassellati (Eds.), {\it Proceedings of the 36th Annual Conference of the Cognitive Science Society} (pp. 1192-1197). Austin, TX: Cognitive Science Society.\\[6pt]
 &\\
&  	\textbf{Popov, V.} \& Petkov, G. (2014). The level of processing affects the magnitude of induced retrograde amnesia. In P. Bello, M. Guarini, M. McShane, \& B. Scassellati (Eds.), {\it Proceedings of the 36th Annual Conference of the Cognitive Science Society} (pp. 2787-2792). Austin, TX: CSS.\\[18pt]
 &\\
\end{longtable}


\subsection*{Journal publications (in Bulgarian)}
\begin{longtable}{p{0.7cm}p{15cm}}
2016 & \textbf{Popov, V.}, Nedelchev, D., Peneva, E., Psederska, E., Georgieva, V., Vasilev, G., \& Vassileva, J. (2016). Psychometric Characteristics of the Bulgarian Version of the Buss-Warren Aggression Questionnaire (BWAQ). {\it Clinical and Consulting Psychology, 4}(30), 37-53.\\[6pt]
&\\
& Nedelchev, D., \textbf{Popov, V.}, Psedarska, E., Bozgunov, K., Vasliev, G., Peneva, E., \& Vassileva, J. (2016). Psychometric Characteristics of the Bulgarian Version of the Wender Utah Rating Scale (WURS-25) for ADHD. {\it Clinical and Consulting Psychology, 8}(2), 3-17 (in Bulgarian)\\[6pt]
&\\
& \textbf{Popov, V.}, Psederska, E., Peneva, E., Bozgunov, K., Vasilev, G., Nedelchev, D., \& Vassileva, J. (2016). Psychometric Characteristics of the Bulgarian Version of the Toronto Alexithymia Scale (TAS-20). {\it Psychological Research, 19}(2), 25-42. (in Bulgarian)\\[6pt]
 &\\
2015 & \textbf{Popov, V.}, Bozgunov, K., Vasilev, G., \& Vassileva, J. (2015). Psychometric Characteristics of the Bulgarian Version of Levenson's Self-report Psychopathy Scale. {\it Bulgarian Journal of Psychology, 1-4}, 253-278.  (in Bulgarian)\\[6pt]
&\\
\end{longtable}


\section*{Open-source software}
\begin{longtable}{p{0.7cm}p{15cm}}
{\it bmm} & Lead author. An R package for Bayesian Measurement Modeling of Visual Working Memory\\	
\end{longtable}

% \subsection*{Abstracts}

% \begin{longtable}{p{0.7cm}p{15cm}}
% 2019 & \textbf{Popov, V.}, So, M., \& Reder, L. (2019). Word frequency affects binding probability not memory precision. {\it Proceedings of the 41st Annual Conference of the Cognitive Science Society} (pp.). Austin, TX: Cognitive Science Society.\\[6pt]
% % &\\
% & \textbf{Popov, V.}, Zhang, Q., Koch, G., Halloway, R. \& Coutanche, M. (2019). The effect on semantic relatedness on associative asymmetry in memory. {\it Proceedings of the 41st Annual Conference of the Cognitive Science Society} (pp.). Austin, TX: Cognitive Science Society.\\[6pt]
% % &\\
% 2018 & \textbf{Popov, V.}, Pavlova, M., \& Hristova, P. (2018). Role vs relational similarity in analogical processing.  {\it Abstracts of the 59th Annual Meeting of the psychonomic Society}. Madison, WI\\[6pt]
% % &\\
% 2017 & \textbf{Popov, V.} \& Reder, L. (2017). Greater discrimination difficulty during visual search training leads to stronger and more distinct representations. {\it Abstracts of the 58th Annual Meeting of the psychonomic Society}. Vancouver, BC, Canada\\[6pt]
% % &\\
% & Reder, L. \& \textbf{Popov, V.} (2017). Manipulation of Stimulus Familiarity: Effects on Working Memory Resources, Performance in Complex Cognitive Tasks, and Knowledge Formation. {\it Abstracts of the 58th Annual Meeting of the psychonomic Society}. Vancouver, BC, Canada\\[6pt]
% % & \\
% & \textbf{Popov, V.}, Pavlova, M., \& Hristova, P. (2017). Semantic Relations Vary in the Strength, Typicality and Accessibility of their Representations in Long-Term Memory. {\it Proceedings of the 4th Analogy Conference}. Paris, France\\[6pt]
% % &\\
% & \textbf{Popov V.}, \& Hristova, P. (2017). The Relational Luring Effect: False Recognition via Relational Similarity. In G. Gunzelmann, A. Howes, T. Tenbrink, \& E. Davelaar (Eds.), {\it Proceedings of the 39th Annual Conference of the Cognitive Science Society} (pp.). Austin, TX: Cognitive Science Society.\\[6pt]
% % &\\
% & Nedelchev, D., \textbf{Popov, V.}, Psederska, E., Bozgunov, K., Vasilev, G., Peneva, E., \& Vassileva, J. (2016). Psychometric characteristics of the Bulgarian version of the Wender Utah Rating Scale for ADHD. {\it 2nd National Congress of Clinical Psychology}. Sofia, Bulgaria. \\[6pt]
% % &\\
% & Bozgunov, K., \textbf{Popov, V.}, Psedarska, E., Vasilev, G., Nedelchev, D., \& Vassileva, J. (2017). Psychometric characteristics of the Bulgarian version of the Toronto Alexithymia Scale (TAS-20). {\it 7th International Congress of Psychologists of Slovenia - Applied Neuropsychology: Between Small and Big Networks}. Kranjska Gora, Slovenia.\\[6pt]
% % &\\
% 2016 & \textbf{Popov, V.} \& Reder, L. (2016). Semantic-episodic interactions during memory retrieval. In {\it Abstracts of the 57th Annual Meeting of the Psychonomic Society}. Boston, Massachusetts\\[6pt]
% % &\\
% & Shen, Z., Reder, L., \textbf{Popov, V.}, \& Delahay, A. (2016). Symbol Familiarity Interacts With Working Memory Demands During Mathematical Problem-Solving. In {\it Abstracts of the 57th Annual Meeting of the Psychonomic Society}. Boston, Massachusetts\\[6pt]
% % &\\
% & Vogt, K., Ibinson, J., Tremel, J., \textbf{Popov, V.}, Reder, L., \& Fiez, J. (2016). Variability in the Effect of Experimental Pain on Longterm Memory During Sedation with Dexmedetomidine \& Midazolam. In {\it Abstracts of the Anestesiology 2016 Annual Meeting}. Chicago, IL\\[6pt]
% % &\\
% & Reder, L. M., Liu, X. L., Keinath, A. \& \textbf{Popov, V.} (2015). Knowledge construction requires bricks, not sand: The critical role of familiar constituents in learning. {\it International Meeting of the Psychonomic Society}. Granada, Spain \\[6pt]
% % &\\
% 2015 &  \textbf{Popov, V.}, Hristova, P. \& Anders, R. (2015). Retrieval of relational information during associative recognition. In {\it Abstracts of the 56th Annual Meeting of the Psychonomic Society}. Chicago, IL\\[6pt]
% % &\\
% & Reder, L. M., Liu, X. L., Keinath, A. \& \textbf{Popov, V.} (2015). Building knowledge requires bricks, not sand: The critical role of familiar constituents in learning. In {\it Abstracts of the 56th Annual Meeting of the Psychonomic Society}. Chicago, IL \\[6pt]
% % &\\
% & \textbf{Popov, V.} (2015).  A spreading-activation model of associates retrieval in a discrete free association task. In N. Taatgen, M. van Vugt, J. Borst, \& K. Mehlhorn (Eds.), {\it Proceedings of the 13th International Conference on Cognitive Modeling}. Groenigen: Rijksuniversiteit Groenigen\\[6pt]
% % &\\
% 2014 & \textbf{Popov, V. }\& Hristova, P. (2014). Unconscious, unintended and efficient analogies in lexical decision under dual-task conditions. In P. Bello, M. Guarini, M. McShane, \& B. Scassellati (Eds.), {\it Proceedings of the 36th Annual Conference of the Cognitive Science Society}. Austin, TX: Cognitive Science Society.\\[6pt]
% % &\\
% & \textbf{Popov, V.} \& Hristova, P. (2014). Priming thematic structure during sentence comprehension in the absence of syntactic repetition. In {\it Abstracts of the VI. Dubrovnik Conference on Cognitive Science.} Dubrovnik, Croatia.\\[6pt]
% % &\\
% & \textbf{Popov, V.} (2014). Malleability of the basic level effect in categorical induction for biological categories. In {\it Abstracts of the VI. Dubrovnik Conference on Cognitive Science}. Dubrovnik, Croatia.\\[6pt]
% % &\\
% 2013 & \textbf{Popov, V.} (2013). Fixation on Failure: Failing to Solve a Problem Hinders Subsequent Problem-Solving Ability. In M. Knauff, M. Pauen, N. Sebanz, \& I. Wachsmuth (Eds.), {\it Proceedings of the 35th Annual Conference of the Cognitive Science Society} (p. 4080). Austin, TX: CSS.\\[6pt]
% \end{longtable}

\section*{Presentations}
\subsection*{Invited talks}
\begin{longtable}{p{0.7cm}p{15cm}}
2023 & One World Cognitive Psychology Seminar Series, Psychonomic Society, online, Sep 2023\\
& {\it If God Handed Us the Ground-Truth Theory of Memory, How Would We Recognize It?}\\[6pt]
2022 & Cognitive Brown Bag Series, Michigan State University, online, November 2022\\ & {\it Resource Depletion and Recovery in Human Memory}\\[6pt] 
& Working Memory Development Lab Seminar, University of Fribourg, April 2022\\ & {\it Frequency effects on memory: A resource-limited theory}\\[6pt] 
2021 & Context and Episodic Memory Symposium, University of Melbourne, online, Dec 2021\\ & {\it Intent Matters: Resolving the Intentional vs Incidental Learning Paradox in Episodic Long-term Memory}\\[6pt]
& Gluschko Dissertation Award Symposium, Cognitive Science Society, online, July 2021\\ & {\it Frequency effects on memory: A resource-limited theory}\\[6pt] 
& Seminar on Memory and Analogy, New Bulgarian University, online, May 2021\\ & {\it Intent Matters: Resolving the Intentional vs Incidental Learning Paradox in Episodic Long-term Memory}\\[6pt]
2019 & Cognitive Seminar, Ohio State University, Columbus, OH, April 2019\\ & {\it Semantic relations play a crucial role in memory, language comprehension and analogical reasoning}\\[6pt]
% &\\
2016 & Workshop on Memory and Skill, Duke University, Durham, North Carolina, April 2016.\\ & {\it Not knowing what we know: A call for a theory-neutral database for empirical results in psychology}.\\[6pt]
\end{longtable}

\subsection*{Departmental talks}
\begin{longtable}{p{0.7cm}p{15cm}}
2019 & Visual Cognition Meeting (VisCog), Carnegie Mellon University, Pittsburgh, Jan 2019.\\ & {\it Practices and pitfalls in inferring neural representations}.\\[6pt]
% &\\
2017 & CNBC Brain Bag series, Center for the Neural Basis of Cognition, Pittsburgh, Nov 2017.\\ & {\it Inferential Pitfalls in Decoding Neural Representations}.\\[6pt]
% &\\
& Pecerption, Action and Learning (PAL) series, Carnegie Mellon University, Pittsburgh, Feb 2017.\\ & {\it The Mortar of Cognition: Semantic relations play a crucial role in memory, language comprehension and analogical reasoning.}\\[6pt]
\end{longtable}

\subsection*{Conference talks}
\begin{longtable}{p{0.7cm}p{15cm}}
2023 & If God Handed Us the Ground-Truth Theory of Memory, How Would We Recognize It? {\it Oral presentation at the 23rd Annual Summer Interdisciplinary Conference (ASIC 2023)}. Kranjska Gora, Slovenia\\[6pt]
2021 & Uncertainty ratings can improve the estimation of memory precision by several orders of magnitude. {\it Oral presentation at the Psychonomic Society 62nd Annual Meeting}. Virtual conference.\\ [6pt]
&\\ 
2019 & Word frequency affects binding probability not memory precision. {\it Oral presentation at the 41st Annual Conference of the Cognitive Science Society}. Montreal, CA.\\[6pt]
 &\\
& The effect of semantic relatedness on associative symmetry in memory. {\it Oral presentation at the 41st Annual Conference of the Cognitive Science Society}. Montreal, CA.\\[6pt]
&\\
& Frequency effects on memory: A resource-limited theory. {\it Oral presentation at the 18th Annual Summer Interdisciplinary Conference (ASIC 2019)}. Seefeld, Austria\\[6pt]
&\\
2017 & Inferential Pitfalls in Decoding Neural Representations. {\it Oral presentation at the 39th Annual Conference of the Cognitive Science Society}. London, UK.\\[6pt]
 &\\
& The Relational Luring Effect: False Recognition via Relational Similarity. {\it Oral presentation at the 39th Annual Conference of the Cognitive Science Society}. London, UK.\\[6pt]
&\\
& Target-to-distractor similarity can help visual search performance. {\it Oral presentation at the 39th Annual Conference of the Cognitive Science Society}. London, UK.\\[6pt]
 &\\
& The Relational Luring Effect: Retrieval of relational information during associative recognition. {\it Oral presentation at the 16th Annual Summer Interdisciplinary Conference (ASIC 2017)}. Interlaken, Switzerland.\\[6pt]
 &\\
2015 & Unintentional relational priming. {\it Oral presentation at the Annual Conference of the Department of Cognitive Science and Psychology}. Kazanlak, Bulgaria\\[6pt]
&\\
2014 &  Priming of relations: Unintended, unconscious and efficient. {\it Oral presentation during the European Campus of Excellence Summer School in Memory and Mind: Perspectives from Philosophy and Neuroscience.} Germany, Bochum: Ruhr University Bochum\\[6pt]
 &\\
& Automatic analogical reasoning underlies structural priming in comprehension of ambiguous sentences. {\it Oral presentation at the 36th Annual Conference of the Cognitive Science Society}. Quebec City, Canada.\\[6pt]
 &\\
& Retrograde amnesia as a function of the level of processing. {\it  Oral presentation at the Annual Conference of the Department of Cognitive Science and Psychology}. Belmeken, Bulgaria: NBU\\[6pt]
 &\\
& Automatic analogies in language comprehension. {\it  Oral presentation at the Annual Conference of the Department of Cognitive Science and Psychology}. Belmeken, Bulgaria: NBU\\[6pt]
 &\\
2013 & Experiencing failure increases problem-solving time in current performance. {\it Oral presentation at the Annual Conference of the Department of Cognitive Science and Psychology}. Ribaritza, Bulgaria\\[6pt]

\end{longtable}

%\subsection*{Conference posters}
%\begin{longtable}{p{0.7cm}p{15cm}}
%2019 & Word frequency affects binding probability not memory precision.. {\it Poster presented at the Annual Meeting of the Psychonomic Society}. Montreal, CA\\[6pt]
% &\\
%& Word frequency affects binding probability not memory precision.. {\it Poster presented at the 2019 Context 
% Episodic Memory Symposium (CEMS)}. Philadelphia, PA\\[6pt]
% &\\
%2018 & Role vs relational similarity in analogical processing. {\it Poster presented at the 59th Annual Meeting of the Cognitive Science Society}. Madison, WI\\[6pt]
% &\\
%&  Fast Memory Integration Facillitated by Schema Consistency. {\it Poster presented at the 59th Annual Meeting of the Cognitive Science Society}. Madison, WI\\[6pt]
% &\\
%2017 & Greater discrimination difficulty during visual search training leads to stronger and more distinct representations. {\it Poster presented at the 58th Annual Meeting of the Cognitive Science Society}. Vancouver, BC, Canada\\[6pt]
% &\\
%& Repetition improves memory by strengthening existing traces: Evidence from paired-associate learning under midazolam. {\it Poster presented at the 39th Annual Conference of the Cognitive Science Society}. London, UK.\\[6pt]
%&\\
%2016 & Semantic-Episodic Interactions During Memory Retrieval. {\it Poster presented at Psychonomic Society's 57th Annual Meeting}. Boston, Massachusetts\\[6pt]
%&\\
%2015 & Retrieval of relational information during associative recognition. {\it Poster presented at Psychonomic Society's 56th Annual Meeting}. Chicago, Illinois\\[6pt]
% &\\
%& Retrieval of relational information during associative recognition. {\it Poster presented at the Center for the Neural Basis of Cognition Annual Retreat}. Pittsburgh, PA.\\[6pt]
% &\\
%& A spreading-activation model of associates retrieval in a discrete free association task. {\it Poster presented at the 13th International Conference on Cognitive Modeling}. Groenigen, Germany\\[6pt]
% &\\
%2014 & The level of processing affects the magnitude of induced retrograde amnesia. {\it Poster presented at the 36th Annual Conference of the Cognitive Science Society}. Quebec City, Canada.\\[6pt]
% &\\
%& Unconscious, unintended and efficient analogies in lexical decision under dual-task conditions. {\it  Poster presented at the 36th Annual Conference of the Cognitive Science Society}. Quebec City, Canada.\\[6pt]
% &\\
%& Malleability of the basic level effect in categorical induction for biological categories. {\it  Poster presented at the VI. Dubrovnik Conference on Cognitive Science}. Dubrovnik, Croatia.\\[6pt]
% &\\
%& Priming thematic structure during sentence comprehension in the absence of syntactic repetition. {\it  Poster presented at the VI. Dubrovnik Conference on Cognitive Science}. Dubrovnik, Croatia.\\[6pt]
% &\\
%2013 & Fixation on Failure: Failing to Solve a Problem Hinders Subsequent Problem-Solving Ability. {\it Poster presented at the 35th Annual Conference of the Cognitive Science Society.} Austin, TX: Cognitive Science Society.\\[6pt]
%
%\end{longtable}

%\section*{Professional activities}
%\subsection*{Memberships}
%
%Cognitive Neuroscience Society, student member\\
%Psychonomic Society, student member\\
%Cognitive Science Society (CSS), student member

%\section*{Departmental and professional service}
%\begin{tabular}{rl}
%	2017 & Graduent student host for invited speaker, Department of Psychology Colloquium series\\ & Carnegie Mellon University\\
%	2016 - 2017 & Graduent student representative, Faculty meetings, Carnegie Mellon University\\
%\end{tabular}

% \section*{Research and work experience}

% \begin{tabular}{rl}
% 	06.2015 - current & Ph.D. student, Memory Lab, Carnegie Mellon University \\
% 	& Director: Lynne M. Reder, Ph.D. \\
% 	&\\	

% 	01.2015 - 08.2015 & Data Analyst, Bulgarian Addictions Institute \\
% 	& Directors: Jasmin Vassileva, Ph.D., \& Georgi Vasilev, M.D. \\
% 	% & \begin{minipage}{0.8\textwidth}
% 	% 		\medskip
% 	% 		I was mainly responsible for data analysis on an international research project funded by grant 2R01DA021421 from the National Institute on Drug Abuse (NIDA) - USA, that aims to explore how different personality and neurocognitive aspects of impulsivity can affect and be affected by drug addiction.
% 	% 	\end{minipage}\\
% 	&\\	

% 	03.2013 - 12.2014 & Researcher and Lab Coordinator, Experimental Psychology Lab, NBU\\
% 	& Director: Assist. Prof. Penka Hristova\\
% 	% & \begin{minipage}{0.8\textwidth}
% 	% 	\medskip
% 	% 	I planed, designed and executed behavioral experiments on analogy-making, constructive memory, concepts and categorization and language comprehension. Other activities included generating stimuli, data analysis, creating e-prime scripts, preparing manuscripts and weekly presentations during seminars.
% 	% \end{minipage}\\

% \end{tabular}

\section*{Teaching experience}

\begin{longtable}{rl}
& \textbf{Mentorship}\\[6pt]
& Berit Barthelmes, UZH, Master's thesis supervisor\\
& Huichao Ji, UZH, Master's internship (now a PhD student at Yale University)\\
& Lorin Schöni, UZH, Master's internship (now a PhD student at ETH Zurich)\\
&\\
& \textbf{Graduate courses at the University of Zürich}\\[6pt]
& How do we measure cognition? (co-teaching with Hannah Dames)\\
& {\it (overall course satisfaction 5.6/6.0; satisfaction with the instructor 5.9/6.0)}\\
& {\it Fall 2022}\\
&\\
& Generating and Testing Explanations of Brain and Behavior: The Good, the Bad, and the Ugly\\
& {\it (overall course satisfaction 5.0/6.0; satisfaction with the instructor 5.5/6.0)}\\
& {\it Spring 2022, Spring 2023}\\
&\\
& Data Science for Psychologists (online) \\
& {\it (overall course satisfaction 4.0/6.0; satisfaction with the instructor 5.0/6.0)}\\
& {\it Spring 2021}\\
&\\
& \textbf{Teaching assistant}\\[6pt]
2019 & Cognitive Research Methods (Instructor: Erik Thiessen)\\
2018 & Introduction to Psychology (Instructor: Danniel Oppenheimer)\\
& Data Science for Psychology \& Neuroscience (Instructor: Timothy Verstynen) \\
2017 & Introduction to Parallel Distributed Processing (Instructor:  David Plaut) \\
&\\
& \textbf{Guest lectures}\\[6pt]
2019 & Computational modelling of memory (Memory. Instructor: Lynne Reder)\\
2018 & Approaches to studying memory (Memory. Instructor: Lynne Reder)\\
2018 & Introduction to classifiers (Data Science for Psychology \& Neuroscience. Instructor: Timothy Verstynen) \\
2015 & Working Memory (Memory. Instructor: Penka Hristova)\\
2014 & Working Memory (Memory and meaning. Instructor: Encho Gerganov)\\
&\\
& \textbf{Non-academic teaching}\\[6pt]
2022 & Alpine Skill-transfer Workshops Coordinator (Academic Alpine Club of Zurich)\\
2019 & Mountaineering School Director (Explorer's Club of Pittsburgh)\\
2019 & Rock Climbing School Instructor (Explorer's Club of Pittsburgh)\\


\end{longtable}

\section*{Editorial and service roles}

\begin{tabular}{rl}
	2021 - current & Consulting Editor, {\it Memory and Cognition}\\
	2021 - current & Symposium co-organizer, {\it Distributed Working Memory Series (DWMS; https://bit.ly/3F27hrS)}
\end{tabular}

\section*{Ad hoc reviewer}

\begin{longtable}{p{14cm}p{0cm}}
- Proceedings of the National Academy of Sciences \\
- NeuroImage \\
- Psychological Review \\
- Psychonomic Bulletin \& Review \\
- Memory and Cognition \\
- Journal of Experimental Psychology: General \\
- Journal of Experimental Psychology: Learning, Memory \& Cognition \\
- Cognitive Psychology\\
- Quarterly Journal of Experimental Psychology \\
- Memory \\
- Attention, Perception \& Psychophysics \\
- Developmental Psychology \\
- Cognition \\
- PLOS One \\
- Behavior Research Methods \\
- Cognitive Science \\
- Cognitive Processing \\
- Brain \& Cognition \\
- Experimental Psychology \\
- Communications Biology \\
- Scientific Reports \\
- Neural Computation \\
- Journal of Psycholinguistic Research\\
- Cognitive Science Society Conference\\
\end{longtable}


%\section*{Technical skills}
%
%I am well versed in a number of advanced statistical approaches. For most of my behavioral work, I employ multilevel modeling, including frequentist mixed-effects linear, logistic and multinomial regressions (e.g., Popov \& Hristova, 2015; Popov, Hristova \& Anders, 2017), as well as Bayesian multi-level models (e.g. Popov et al., 2017; Popov, Marevic, Rummel \& Reder, 2019). During my pre-graduate job as a Data Scientist, I employed various psychometric methods such as factor analysis, principle components analysis, ROC analysis, and others, for testing and validating a number of personality scales (e.g. Popov et al., 2016). As part of my statistical training, I have assisted teaching a graduate-level course in Data Science for Psychology and Neuroscience (CMU-85732), and a graduate-level course in Parallel Distributed Processing (CMU-85719). \\
%
%In my computational modelling work, I have employed connectionist modelling (Popov \& Reder, 2017), hierarchical drift-diffusion modelling (Popov, Hristova \& Anders, 2017), mixture modelling of continuous report data (Popov, So \& Reder, 2019), and the Source of Activation Confusion process model of memory (Popov \& Reder, in press; Popov, Marevic, Rummel \& Reder, 2019). I have also used simulations of neural encoding models to highlight some inferential pitfalls in cognitive neuroscience (Popov, Ostarek \& Tenison, 2018).
%
%\section*{Open Science Commitment}
%
%I have a strong commitment to open science principles that make my work more transparent and reproducible. I perform all statistical analysis via R and/or Python scripts and I routinely share the experimental software, stimuli, analysis code and raw data for my projects on my website (http://venpopov.com/2017/01/06/publications/), my github page (https://github.com/venpopov?tab=repositories), and the Open Science Framework. I often employ preregistration and planned replications of surprising findings (e.g., Popov, Marevic, Rummel \& Reder, 2019; Popov, Zhang, Koch, Calloway \& Coutanche, 2019).


% \section*{Additional education and professional training}
% European Campus of Excellence Summer School, Memory and Mind: Perspectives from Philosophy and Neuroscience. Germany, Bochum: Ruhr University Bochum, 7-21 September (2014)\smallskip\\
% 20th International Summer School in Cognitive Science. Sofia, Bulgaria: NBU (2013)\smallskip\\
% 19th International Summer School in Cognitive Science. Sofia, Bulgaria: NBU (2012)

% \section*{Research interests}
% Neural mechanisms of memory formation, consolidation and retrieval. Representation of semantic knowledge in the brain. Computational modeling of neural and psychological mechanisms. Other topics: brain-computer interfaces, relational reasoning, concepts and categorization, problem-solving, cognitive modeling, connectionism, philosophy of science

% \section*{Other information}
% \subsection*{Technical competencies}
% \begin{itemize}
% \item Data analysis, modeling and visualization with R (lme4, ggplot, R2jags, rube, RMarkdown, etc)
% \item Matlab (SPM12), Python (nipype), Freesurfer
% \item SPSS, LaTeX, Advanced MS OFFICE skills 
% \item E-prime 2.0, OpenSesame, Python
% \item Basic knowledge in PHP, HTML, MYSQL, Photoshop
% \end{itemize}
% \subsection*{Language competencies}
% \begin{itemize}
% \item Bulgarian (mother tongue) | English | Some German
% \end{itemize}
% \subsection*{Graduate record examination (GRE)}
% \begin{itemize}
% \item VR: 167/170; 97 percentile rank; QR: 170/170; 98 percentile rank, AW: 5.0/6.0; 93 percentile rank
% \end{itemize}








\bigskip

% Footer
%\begin{center}
%  \begin{footnotesize}
%    Last updated: \today \\
%    \href{\footerlink}{\texttt{\footerlink}}
%  \end{footnotesize}
%\end{center}

\end{document}
